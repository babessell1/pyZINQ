%%
%% Copyright 2022 OXFORD UNIVERSITY PRESS
%%
%% This file is part of the 'oup-authoring-template Bundle'.
%% ---------------------------------------------
%%
%% It may be distributed under the conditions of the LaTeX Project Public
%% License, either version 1.2 of this license or (at your option) any
%% later version.  The latest version of this license is in
%%    http://www.latex-project.org/lppl.txt
%% and version 1.2 or later is part of all distributions of LaTeX
%% version 1999/12/01 or later.
%%
%% The list of all files belonging to the 'oup-authoring-template Bundle' is
%% given in the file `manifest.txt'.
%%
%% Template article for OXFORD UNIVERSITY PRESS's document class `oup-authoring-template'
%% with bibliographic references
%%

%%%CONTEMPORARY%%%
\documentclass[unnumsec,webpdf,contemporary,large]{oup-authoring-template}%
%\documentclass[unnumsec,webpdf,contemporary,large,namedate]{oup-authoring-template}% uncomment this line for author year citations and comment the above
%\documentclass[unnumsec,webpdf,contemporary,medium]{oup-authoring-template}
%\documentclass[unnumsec,webpdf,contemporary,small]{oup-authoring-template}

%%%MODERN%%%
%\documentclass[unnumsec,webpdf,modern,large]{oup-authoring-template}
%\documentclass[unnumsec,webpdf,modern,large,namedate]{oup-authoring-template}% uncomment this line for author year citations and comment the above
%\documentclass[unnumsec,webpdf,modern,medium]{oup-authoring-template}
%\documentclass[unnumsec,webpdf,modern,small]{oup-authoring-template}

%%%TRADITIONAL%%%
%\documentclass[unnumsec,webpdf,traditional,large]{oup-authoring-template}
%\documentclass[unnumsec,webpdf,traditional,large,namedate]{oup-authoring-template}% uncomment this line for author year citations and comment the above
%\documentclass[unnumsec,namedate,webpdf,traditional,medium]{oup-authoring-template}
%\documentclass[namedate,webpdf,traditional,small]{oup-authoring-template}

%\onecolumn % for one column layouts

%\usepackage{showframe}

\graphicspath{{Fig/}}

% line numbers
%\usepackage[mathlines, switch]{lineno}
%\usepackage[right]{lineno}

\theoremstyle{thmstyleone}%
\newtheorem{theorem}{Theorem}%  meant for continuous numbers
%%\newtheorem{theorem}{Theorem}[section]% meant for sectionwise numbers
%% optional argument [theorem] produces theorem numbering sequence instead of independent numbers for Proposition
\newtheorem{proposition}[theorem]{Proposition}%
%%\newtheorem{proposition}{Proposition}% to get separate numbers for theorem and proposition etc.
\theoremstyle{thmstyletwo}%
\newtheorem{example}{Example}%
\newtheorem{remark}{Remark}%
\theoremstyle{thmstylethree}%
\newtheorem{definition}{Definition}

\begin{document}

\journaltitle{BIOSTAT 666 Final Project}
\DOI{xxxxxxxxxx}
\copyrightyear{2024}
\pubyear{2024}
\access{Submitted: 22 April 2023}
\appnotes{Project Writeup}

\firstpage{1}

%\subtitle{Subject Section}

\title[Short Article Title]{Controlling for covariates in \emph{de novo} short tandem repeat expansion association tests with ZINQ}

\author[1]{Brandt Bessell}
\author[1]{Xiaomeng Du}

\authormark{Author Name et al.}

\address[1]{\orgdiv{Department of Computational Medicine and Bioinformatics}, \orgname{University of Michigan}}

%\editor{Associate Editor: Name}

%\abstract{
%\textbf{Motivation:} .\\
%\textbf{Results:} .\\
%\textbf{Availability:} .\\
%\textbf{Contact:} \href{name@email.com}{name@email.com}\\
%\textbf{Supplementary information:} Supplementary data are available at \textit{Journal Name}
%online.}

\abstract{Abstracts must be able to stand alone and so cannot contain citations to
the paper's references, equations, etc. An abstract must consist of a single
paragraph and be concise. Because of online formatting, abstracts must appear
as plain as possible.}
\keywords{keyword1, Keyword2, Keyword3, Keyword4}

% \boxedtext{
% \begin{itemize}
% \item Key boxed text here.
% \item Key boxed text here.
% \item Key boxed text here.
% \end{itemize}}

\maketitle


\section{Introduction}

Short tandem repeats (STRs) are repeated nucleotide motifs 2-9 base pairs long and make up
 ~3-5\% of the human genome\cite{gymrek_genomic_2017}. STRs can contract and expand through
several mechanisms, most commonly slipped-strand mispairing. As a result of their inherent
instability, their size and sometimes motif composition differ not only between individuals, 
but within the somatic cells of individuals\cite{cortes-ciriano_molecular_2017}. About 60 highly
penetrant STRs have been identified to cause human disease\cite{paulson_repeat_2018}, but as
STR's native and regulatory functions are being increasingly considered in modern study\cite{wright_native_2023},
the potential for association studies to uncover the more systematic roles they play in human
phenotypes are yet to be realized due to many factors. The most important being that repetitive
elements larger than the read length can not be accurately measured at high-throughput due to the
inability to uniquely map them to the reference genome with short reads. Further, larger STRs 
are harder to amplify with PCR due to primer stuttering which can drastically reduce the signal
from larger expansions of interest.\cite{dashnow_strling_2022} Association tests either rely on
STRs that are smaller than the read length6, or they use tools designed to approximate the repeat
expansion size through estimates using read/kmer counting methods that require catalogs of known
polymorphic STR loci\cite{margoliash_polymorphic_nodate}. Association tests using these estimates
often perform an inverse rank-based normal transformation of STR size in order to assume an 
approximate distribution that allows for consideration of covariates, or nonparametric tests with
no consideration for covariates8. Modern STR calling algorithms designed to identify larger 
expansions in potentially non-cataloged repeat expansion loci, however, will only report
expansions that are confidently larger than reference regions of low Shannon entropy\cite{dashnow_strling_2022},
\cite{dolzhenko_expansionhunter_2020}. This often leads to left-censored data with inflated 
zeros since expansion size estimates are reported relative to the reference (Figure 1).
Further, the heterogeneity of larger expansion size distributions makes it difficult to
assume distributions for regression in this context. As a result of this, even modern 
case-control analysis software using de novo STR expansion estimates use nonparametric tests
that cannot control for population structure or other covariates\cite{dolzhenko_expansionhunter_2020}.

\subsection{Zero inflation in microbiome studies}\label{subsec1}

Evaluation of the association between microbiome RNA abundance and disease is a common task in
microbiome studies that similarly faces the issue of zero inflation. Many protocols for 
handling zero inflation in microbiome studies have been proposed. A method called ZINQ\cite{ling_powerful_2021}
(Zero-Inflated Quantile Approach) has been specifically designed to address the following limitations
and important analytical details for microbiome RNA abundance data that overlap with those in
\emph{de novo} STR association studies which are outlined below.

\begin{enumerate}
    \item Like de novo STR expansion estimates, distributions of microbiome RNA abundance data are heterogeneous which means association tests should be non-parametric.
    \item Like de novo STR expansion estimates, distributions are often left-censored and/or zero-inflated, so the method must be robust enough to handle such cases, without overemphasizing the differences in distributions based solely on the number of zeros in the cases vs controls.
    \item Like de novo STR expansion estimates, microbiome expression data can be severely impacted by batch effects and other technical and biological variation. In the case of STR expansion estimates we further have to further consider population structure, sample tissue, GC-bias, sequencing center, etc.
    \item Like with bacterial taxa, short tandem repeat-related pathogenicity is more likely to manifest in higher quantiles.
  \end{enumerate}

\subsection{Application of ZINQ in \textit{de novo} STR expansion association studies}\label{subsec1}

Our project explores the potential utility of ZINQ in the context of de novo STR expansion 
association tests and adapts the method to allow for ensembling STR expansion estimates from
multiple  algorithms that apply different techniques with different assumptions. We use the 
ZINQ method to test for association between STR expansion estimates from STRling and
ExpansionHunter \emph{de novo} and Alzheimer's disease in 24,859 individuals in the
NIAGADs dataset and compare the inflation to a Wilcoxon Rank-sum test. We augment the Cauchy
Combination Test (CCT) step in the ZINQ method to combine the marginal p-values from 
two different methods to further denoise and improve the signal from our more likely associations.
Though we use the R implementation of ZINQ for assesment of its utility for our research question,
as a project deliverable we have reimplemented a publically available python implementation of ZINQ.
This is to allow us to have greater control over how the application interfaces with our analysis
pipeline and our abilitiy to expand it's features in the future, and as a learning tool to
enrich our statistical and programmatic understanding of the method.

\section{Methods}

\subsection{Zero-Inflated Quantile Approach (ZINQ)}\label{subsec2}

overview of ZINQ...

\paragraph{Firth's Logistic Regression with Likelihood Ratio Test}

overview of Firth's logistic regression... with equations

\paragraph{Quantile Regression with Rank Score Test}

overview of quantile regression... with equations

\paragraph{Cauchy Combination Test}

overview of Cauchy combination test... with equations

\subsection{Selection of covariates}

...

\subsection{Examples of How to make Equations}\label{subsec3}


Equations in \LaTeX{} can either be inline or set as display equations. For
inline equations use the \verb+$...$+ commands. Eg: the equation
$H\psi = E \psi$ is written via the command \verb+$H \psi = E \psi$+.

For display equations (with auto generated equation numbers)
one can use the equation or eqnarray environments:
\begin{equation}
\|\tilde{X}(k)\|^2 \leq\frac{\sum\limits_{i=1}^{p}\left\|\tilde{Y}_i(k)\right\|^2+\sum\limits_{j=1}^{q}\left\|\tilde{Z}_j(k)\right\|^2 }{p+q},\label{eq1}
\end{equation}
where,
\begin{align}
D_\mu &=  \partial_\mu - ig \frac{\lambda^a}{2} A^a_\mu \nonumber \\
F^a_{\mu\nu} &= \partial_\mu A^a_\nu - \partial_\nu A^a_\mu + g f^{abc} A^b_\mu A^a_\nu.\label{eq2}
\end{align}
Notice the use of \verb+\nonumber+ in the align environment at the end
of each line, except the last, so as not to produce equation numbers on
lines where no equation numbers are required. The \verb+\label{}+ command
should only be used at the last line of an align environment where
\verb+\nonumber+ is not used.
\begin{equation}
Y_\infty = \left( \frac{m}{\textrm{GeV}} \right)^{-3}
    \left[ 1 + \frac{3 \ln(m/\textrm{GeV})}{15}
    + \frac{\ln(c_2/5)}{15} \right].
\end{equation}
The class file also supports the use of \verb+\mathbb{}+, \verb+\mathscr{}+ and
\verb+\mathcal{}+ commands. As such \verb+\mathbb{R}+, \verb+\mathscr{R}+
and \verb+\mathcal{R}+ produces $\mathbb{R}$, $\mathscr{R}$ and $\mathcal{R}$
respectively (refer Subsubsection~\ref{subsubsec3}).


Lorem ipsum dolor sit amet, consectetur adipiscing elit, sed do
eiusmod tempor incididunt ut labore et dolore magna aliqua. Ut enim ad minim veniam, quis nostrud exercitation ullamco laboris nisi ut aliquip ex ea commodo consequat. Duis aute irure dolor in reprehenderit in voluptate velit esse cillum dolore eu fugiat nulla pariatur. Excepteur sint occaecat cupidatat non proident, sunt in culpa qui officia deserunt mollit anim id est laborum. Lorem ipsum dolor sit amet, consectetur adipiscing elit, sed do
eiusmod tempor incididunt ut labore et dolore magna aliqua. Ut enim ad minim veniam, quis nostrud exercitation ullamco laboris nisi ut aliquip ex ea commodo consequat. Duis aute irure dolor in reprehenderit in voluptate velit esse cillum dolore eu fugiat nulla pariatur. Excepteur sint occaecat cupidatat non proident, sunt in culpa qui officia deserunt mollit anim id est laborum. 
Lorem ipsum dolor sit amet, consectetur adipiscing elit, sed do
eiusmod tempor incididunt ut labore et dolore magna aliqua. Ut enim ad minim veniam, quis nostrud exercitation ullamco laboris nisi ut aliquip ex ea commodo consequat. 


\section{Examples of how to use tables}\label{sec5}

Tables can be inserted via the normal table and tabular environment. To put
footnotes inside tables one has to Lorem ipsum dolor sit amet, consectetur adipiscing elit, sed do eiusmod tempor incididunt ut labore et dolore magna aliqua. Ut enim ad minim veniam, quis nostrud exercitation ullamco laboris nisi ut aliquip ex ea commodo consequat. Duis aute irure dolor in reprehenderit in voluptate velit esse cillum dolore eu fugiat nulla pariatur. Excepteur sint occaecat cupidatat non proident, sunt in culpa qui officia deserunt mollit anim id est laborum. use the additional ``tablenotes" environment
enclosing the tabular environment. The footnote appears just below the table
itself (refer Tables~\ref{tab1} and \ref{tab2}).


\begin{verbatim}
\begin{table}[t]
\begin{center}
\begin{minipage}{<width>}
\caption{<table-caption>\label{<table-label>}}%
\begin{tabular}{@{}llll@{}}
\toprule
column 1 & column 2 & column 3 & column 4\\
\midrule
row 1 & data 1 & data 2          & data 3 \\
row 2 & data 4 & data 5$^{1}$ & data 6 \\
row 3 & data 7 & data 8      & data 9$^{2}$\\
\botrule
\end{tabular}
\begin{tablenotes}%
\item Source: Example for source.
\item[$^{1}$] Example for a 1st table footnote.
\item[$^{2}$] Example for a 2nd table footnote.
\end{tablenotes}
\end{minipage}
\end{center}
\end{table}
\end{verbatim}

\section{Results}\label{sec6}

As per display \LaTeX\ standards one has to use eps images for \verb+latex+ compilation and \verb+pdf/jpg/png+ images for
\verb+pdflatex+ compilation. This is one of the major differences between \verb+latex+
and \verb+pdflatex+. The images should be single-page documents. The command for inserting images
for \verb+latex+ and \verb+pdflatex+ can be generalized. The package used to insert images in \verb+latex/pdflatex+ is the
graphicx package. Figures can be inserted via the normal figure environment as shown in the below example:


\begin{figure}[!t]%
\centering
{\color{black!20}\rule{213pt}{37pt}}
\caption{This is a widefig. This is an example of a long caption this is an example of a long caption  this is an example of a long caption this is an example of a long caption}\label{fig1}
\end{figure}

\begin{figure*}[!t]%
\centering
{\color{black!20}\rule{438pt}{74pt}}
\caption{This is a widefig. This is an example of a long caption this is an example of a long caption  this is an example of a long caption this is an example of a long caption}\label{fig2}
\end{figure*}


\begin{verbatim}
\begin{figure}[t]
        \centering\includegraphics{<eps-file>}
        \caption{<figure-caption>}
        \label{<figure-label>}
\end{figure}
\end{verbatim}

Test text here.

For sample purposes, we have included the width of images in the
optional argument of \verb+\includegraphics+ tag. Please ignore this.
Lengthy figures which do not fit within textwidth should be set in rotated mode. For rotated figures, we need to use \verb+\begin{sidewaysfigure}+ \verb+...+ \verb+\end{sidewaysfigure}+ instead of the \verb+\begin{figure}+ \verb+...+ \verb+\end{figure}+ environment.

\begin{sidewaysfigure}%
\centering
{\color{black!20}\rule{610pt}{102pt}}
\caption{This is an example for a sideways figure. This is an example of a long caption this is an example of a long caption  this is an example of a long caption this is an example of a long caption}\label{fig3}
\end{sidewaysfigure}



\section{Algorithms, Program codes and Listings}\label{sec7}

Packages \verb+algorithm+, \verb+algorithmicx+ and \verb+algpseudocode+ are used for setting algorithms in latex.
For this, one has to use the below format:


\begin{verbatim}
\begin{algorithm}
\caption{<alg-caption>}\label{<alg-label>}
\begin{algorithmic}[1]
. . .
\end{algorithmic}
\end{algorithm}
\end{verbatim}


You may need to refer to the above-listed package documentations for more details before setting an \verb+algorithm+ environment.
To set program codes, one has to use the \verb+program+ package. We need to use the \verb+\begin{program}+ \verb+...+
\verb+\end{program}+ environment to set program codes.

\begin{algorithm}[!t]
\caption{Calculate $y = x^n$}\label{algo1}
\begin{algorithmic}[1]
\Require $n \geq 0 \vee x \neq 0$
\Ensure $y = x^n$
\State $y \Leftarrow 1$
\If{$n < 0$}
        \State $X \Leftarrow 1 / x$
        \State $N \Leftarrow -n$
\Else
        \State $X \Leftarrow x$
        \State $N \Leftarrow n$
\EndIf
\While{$N \neq 0$}
        \If{$N$ is even}
            \State $X \Leftarrow X \times X$
            \State $N \Leftarrow N / 2$
        \Else[$N$ is odd]
            \State $y \Leftarrow y \times X$
            \State $N \Leftarrow N - 1$
        \EndIf
\EndWhile
\end{algorithmic}
\end{algorithm}

Similarly, for \verb+listings+, one has to use the \verb+listings+ package. The \verb+\begin{lstlisting}+ \verb+...+ \verb+\end{lstlisting}+ environment is used to set environments similar to the \verb+verbatim+ environment. Refer to the \verb+lstlisting+ package documentation for more details on this.


\begin{minipage}{\hsize}%
\lstset{language=Pascal}% Set your language (you can change the language for each code-block optionally)
\begin{lstlisting}[frame=single,framexleftmargin=-1pt,framexrightmargin=-17pt,framesep=12pt,linewidth=0.98\textwidth]
for i:=maxint to 0 do
begin
{ do nothing }
end;
Write('Case insensitive ');
Write('Pascal keywords.');
\end{lstlisting}
\end{minipage}



\section{Cross referencing}\label{sec8}

Environments such as figure, table, equation, and align can have a label
declared via the \verb+\label{#label}+ command. For figures and table
environments one should use the \verb+\label{}+ command inside or just
below the \verb+\caption{}+ command.  One can then use the
\verb+\ref{#label}+ command to cross-reference them. As an example, consider
the label declared for Figure \ref{fig1} which is
\verb+\label{fig1}+. To cross-reference it, use the command
\verb+ Figure \ref{fig1}+, for which it comes up as
``Figure~\ref{fig1}".

\subsection{Details on reference citations}\label{subsec3}

With standard numerical .bst files, only numerical citations are possible.
With an author-year .bst file, both numerical and author-year citations are possible.

If author-year citations are selected, \verb+\bibitem+ must have one of the following forms:


{\footnotesize%
\begin{verbatim}
\bibitem[Jones et al.(1990)]{key}...
\bibitem[Jones et al.(1990)Jones,
                Baker, and Williams]{key}...
\bibitem[Jones et al., 1990]{key}...
\bibitem[\protect\citeauthoryear{Jones,
                Baker, and Williams}
                {Jones et al.}{1990}]{key}...
\bibitem[\protect\citeauthoryear{Jones et al.}
                {1990}]{key}...
\bibitem[\protect\astroncite{Jones et al.}
                {1990}]{key}...
\bibitem[\protect\citename{Jones et al., }
                1990]{key}...
\harvarditem[Jones et al.]{Jones, Baker, and
                Williams}{1990}{key}...
\end{verbatim}}


This is either to be made up manually, or to be generated by an
appropriate .bst file with BibTeX. Then,


{%
\begin{verbatim}
                    Author-year mode
                        || Numerical mode
\citet{key} ==>>  Jones et al. (1990)
                        || Jones et al. [21]
\citep{key} ==>> (Jones et al., 1990) || [21]
\end{verbatim}}


\noindent
Multiple citations as normal:


{%
\begin{verbatim}
\citep{key1,key2} ==> (Jones et al., 1990;
                         Smith, 1989)||[21,24]
        or (Jones et al., 1990, 1991)||[21,24]
        or (Jones et al., 1990a,b)   ||[21,24]
\end{verbatim}}


\noindent
\verb+\cite{key}+ is the equivalent of \verb+\citet{key}+ in author-year mode
and  of \verb+\citep{key}+ in numerical mode. Full author lists may be forced with
\verb+\citet*+ or \verb+\citep*+, e.g.


{%
\begin{verbatim}
\citep*{key} ==>> (Jones, Baker, and Mark, 1990)
\end{verbatim}}


\noindent
Optional notes as:


{%
\begin{verbatim}
\citep[chap. 2]{key}     ==>>
        (Jones et al., 1990, chap. 2)
\citep[e.g.,][]{key}     ==>>
        (e.g., Jones et al., 1990)
\citep[see][pg. 34]{key} ==>>
        (see Jones et al., 1990, pg. 34)
\end{verbatim}}


\noindent
(Note: in standard LaTeX, only one note is allowed, after the ref.
Here, one note is like the standard, two make pre- and post-notes.)


{%
\begin{verbatim}
\citealt{key}   ==>> Jones et al. 1990
\citealt*{key}  ==>> Jones, Baker, and
                        Williams 1990
\citealp{key}   ==>> Jones et al., 1990
\citealp*{key}  ==>> Jones, Baker, and
                        Williams, 1990
\end{verbatim}}


\noindent
Additional citation possibilities (both author-year and numerical modes):


{%
\begin{verbatim}
\citeauthor{key}       ==>> Jones et al.
\citeauthor*{key}      ==>> Jones, Baker, and
                                Williams
\citeyear{key}         ==>> 1990
\citeyearpar{key}      ==>> (1990)
\citetext{priv. comm.} ==>> (priv. comm.)
\citenum{key}          ==>> 11 [non-superscripted]
\end{verbatim}}


\noindent
Note: full author lists depend on whether the bib style supports them;
if not, the abbreviated list is printed even when full is requested.

\noindent
For names like della Robbia at the start of a sentence, use


{%
\begin{verbatim}
\Citet{dRob98}      ==>> Della Robbia (1998)
\Citep{dRob98}      ==>> (Della Robbia, 1998)
\Citeauthor{dRob98} ==>> Della Robbia
\end{verbatim}}

\section{Supplementary Figures}

% this will pretty much just be our PCA plots


\section{Discussion}

Some Conclusions here.



\bibliographystyle{plain}
\bibliography{reference.bib}
\end{document}
